\documentclass[oneside,bibliography=totocnumbered,BCOR=5mm]{scrbook}

\usepackage[ngerman]{babel}

\usepackage{marvosym}
\usepackage{csquotes}
\usepackage{hyperref}

\usepackage[
backend=biber,
style=numeric,
citestyle=authoryear,
autocite=footnote
]{biblatex}
\addbibresource{bibliography.bib}
\addbibresource{extra.bib}

\usepackage{graphicx}
\graphicspath{ {images/} }

% \usepackage[hyphens]{url}

\usepackage[sc]{mathpazo} % Use the Palatino font
\usepackage[T1]{fontenc} % Use 8-bit encoding that has 256 glyphs
\linespread{1.05} % Line spacing - Palatino needs more space between lines
\usepackage{microtype} % Slightly tweak font spacing for aesthetics

\usepackage[hmarginratio=1:1,top=32mm,columnsep=20pt]{geometry} % Document margins
\usepackage[hang, small,labelfont=bf,up,textfont=it,up]{caption} % Custom captions under/above floats in tables or figures
\usepackage{booktabs} % Horizontal rules in tables
\usepackage{lettrine} % The lettrine is the first enlarged letter at the beginning of the text
\usepackage{enumitem} % Customized lists
\setlist[itemize]{noitemsep} % Make itemize lists more compact

% \usepackage{abstract} % Allows abstract customization
% \renewcommand{\abstractnamefont}{\normalfont\bfseries} % Set the "Abstract" text to bold
% \renewcommand{\abstracttextfont}{\normalfont\small\itshape} % Set the abstract itself to small italic text

% \usepackage{titlesec} % Allows customization of titles
%\renewcommand\thesection{\Roman{section}} % Roman numerals for the sections
%\renewcommand\thesubsection{\roman{subsection}} % roman numerals for subsections
% \titleformat{\section}[block]{\large\scshape\centering}{\thesection.}{1em}{} % Change the look of the section titles
% \titleformat{\subsection}[block]{\large}{\thesubsection.}{1em}{} % Change the look of the section titles

\usepackage{fancyhdr} % Headers and footers
\pagestyle{fancy} % All pages have headers and footers
\fancyhead{} % Blank out the default header
\fancyfoot{} % Blank out the default footer
% \fancyhead[C]{Ethics in Progress (EiP) $\bullet$ 2019 } % Custom header text
% \fancyfoot[RO,LE]{\thepage} % Custom footer text

\usepackage{titling} % Customizing the title section

\usepackage{listings}
\usepackage{color}
\definecolor{mygreen}{rgb}{0,0.6,0}
\definecolor{mygray}{rgb}{0.5,0.5,0.5}
\definecolor{mymauve}{rgb}{0.58,0,0.82}

\lstset{
  backgroundcolor=\color{white},   % choose the background color; you must add \usepackage{color} or \usepackage{xcolor}; should come as last argument
  basicstyle=\footnotesize,        % the size of the fonts that are used for the code
  breakatwhitespace=false,         % sets if automatic breaks should only happen at whitespace
  breaklines=true,                 % sets automatic line breaking
  captionpos=b,                    % sets the caption-position to bottom
  commentstyle=\color{mygreen},    % comment style
  deletekeywords={...},            % if you want to delete keywords from the given language
  escapeinside={\%*}{*)},          % if you want to add LaTeX within your code
  extendedchars=true,              % lets you use non-ASCII characters; for 8-bits encodings only, does not work with UTF-8
  firstnumber=1,                % start line enumeration with line 1000
  frame=single,	                   % adds a frame around the code
  keepspaces=true,                 % keeps spaces in text, useful for keeping indentation of code (possibly needs columns=flexible)
  keywordstyle=\color{blue},       % keyword style
  language=Octave,                 % the language of the code
  morekeywords={*,...},            % if you want to add more keywords to the set
  numbers=left,                    % where to put the line-numbers; possible values are (none, left, right)
  numbersep=5pt,                   % how far the line-numbers are from the code
  numberstyle=\tiny\color{mygray}, % the style that is used for the line-numbers
  rulecolor=\color{black},         % if not set, the frame-color may be changed on line-breaks within not-black text (e.g. comments (green here))
  showspaces=false,                % show spaces everywhere adding particular underscores; it overrides 'showstringspaces'
  showstringspaces=false,          % underline spaces within strings only
  showtabs=false,                  % show tabs within strings adding particular underscores
  stepnumber=1,                    % the step between two line-numbers. If it's 1, each line will be numbered
  stringstyle=\color{mymauve},     % string literal style
  tabsize=2,	                   % sets default tabsize to 2 spaces
  title=\lstname                   % show the filename of files included with \lstinputlisting; also try caption instead of title
}

\definecolor{mygrey}{RGB}{28,28,28}
\newcommand{\codeinline}[1]{\colorbox{mygray}{\csname lstinline\endcsname!#1!}}

\begin{document}

% Titelseite
% \pagestyle{empty}       % keine Seitennummer
\begin{titlepage}
\begin{center}
\includegraphics{htw-logo.jpg}
\linebreak[4]
\linebreak[4]
\linebreak[4]
\linebreak[4]
\textit{\large Entwicklung und Evaluation von Methoden zur Absenkung der Nutzungsschwelle von Kommandozeilen-Interfaces}
\linebreak[4]
\linebreak[4]
\linebreak[4]
Abschlussarbeit
\linebreak[4]
\linebreak[4]
zur Erlangung des akademischen Grades:
\linebreak[4]
\linebreak[4]
\textbf{Bachelor of Science (B.Sc.)}
\linebreak[4]
\linebreak[4]
an der
\linebreak[4]
\linebreak[4]
Hochschule f\"ur Technik und Wirtschaft (HTW) Berlin
\linebreak[4]
Fachbereich 4: Informatik, Kommunikation und Wirtschaft
\linebreak[4]
Studiengang \textit{Angewandte Informatik}
\linebreak[4]
\linebreak[4]
\linebreak[4]
1. Gutachter: Titel akademischer Grad Vorname Nachname\linebreak[4]
2. Gutachter: B.Sc. Moritz Wachter\linebreak[4]
\linebreak[4]
\linebreak[4]
\linebreak[4]
\linebreak[4]
Eingereicht von Jonathan Neidel [573619]
\linebreak[4]
\linebreak[4]
\linebreak[4]
\linebreak[4]
Datum

\end{center}
\end{titlepage}
\newpage

\thispagestyle{empty}
\vspace*{2.2cm}
\noindent
{\Huge Danksagung}\\
\vspace*{1.6cm} \\

% Kopfzeilen (automatisch erzeugt)
%\pagestyle{headings}
[Text der Danksagung]

% Seite mit Abstracts
\newpage
\thispagestyle{empty}
\section*{Zusammenfassung}
[Text der Zusammenfassung]

\section*{Abstract}
[Summary of the thesis]


\clearpage
%Seite 1
\pagenumbering{roman}
%\setcounter{page}{1}

\tableofcontents
.
\newpage

\pagenumbering{arabic}
% \setcounter{page}{1}   % setzt Seitenzaehlung auf 1

% \fbox{\parbox{\linewidth}{
% }}

% \\
% \linebreak[4]

% \footnote{Erg\"anzende Informationen k\"onnen Sie auch in eine Fu"snote auslagern. Hier wird die Fu"snote dazu genutzt, um Ihnen bei Interesse am Thema Zitation vertiefende Quellen (z.B. \autocite{balzert2011} oder \autocite{franck2013}) anzubieten.}

% \begin{table}
% \caption{\"Ubersicht: Untersuchte Steinl\"ause}
% \centering
% \begin{tabular}{llr}
% \toprule
% \multicolumn{2}{c}{Untersuchte Objekte mit Lokation des Habitats} \\
% \cmidrule(r){1-2}
% ID (nickname) & Ort & Gr\"o"se/L\"ange (in mm) \\
% \midrule
% 1 (Rosalinde) & Berlin, Mauerpark & $1.4$ \\
% 2 (Devil in disguise) & Brandenburg, BER-Airport & $2.8$ \\
% 3 (Hannes) & Berlin, Olympia-Stadion & $2.1$ \\
% 4 (Her Majesty) & Berlin, Humboldt-Forum & $2.0$ \\
% \bottomrule
% \end{tabular}
% \end{table}

% \begin{lstlisting}[caption={Ein Beispiel: Hello World (Scala)}]
% object HelloWorld {
%   def main(args: Array[String]): Unit = {
%     println("Hello, world!")
%   }
% }
% \end{lstlisting}

\chapter{Einleitung}
\section{Hintergrund der Arbeit}
% [Beschreibung des groben Kontextes der Arbeit; im Detail sollten Sie dies im Grundlagenteil darstellen]

Die Kommandozeile und darauf basierende Applikationen bergen das Potential für Produktivitätssteigerungen im Vergleich zu GUI Applikationen, vorrausgesetzt der Nutzer weiß mit dieser umzugehen. % TODO: Citation needed


\section{Problem- und Zielstellung (Scope)}
% [Beschreibung der Problemstellung sowie der sich daraus ergebenden Teilprobleme,-ziele und Forschungsfrage(n), welche Sie mit Ihrer Arbeit addressieren]

Die Kommandozeile und seine Applikationen sind für Personen welche mit dieser Umgebung nicht vertraut sind schwer benutzbar. % TODO: Citation needed
Einflussfaktoren dafür sind:

\begin{enumerate}
  \item Fehlendes Wissen über das Ökosystem (wie werden Applikationen gestartet, wie findet man Hilfe, Verständnis grundsätzlicher Werkzeuge fehlt)
  \item Applikationen sind nicht für Neulinge konzipiert
\end{enumerate}

Problemstellung dieser Arbeit soll das zweite der gelisteten Probleme sein: Das Zusammentragen und Evaluieren der Faktoren welche Kommandozeilen Applikation für Neulinge zugänglich machen. Oder konkreter als Forschungsfrage definiert:

\bigbreak

\fbox{\parbox{\linewidth}{
Welche Faktoren sind für die Konstruktion einer Command-Line basierten App zu
beachten um diese für Personen, welche nicht mit dem Terminal vertraut sind,
zugänglich zu machen?
}}

\section{Aufbau der Arbeit}
% [Beschreibung des Aufbaus der Arbeit]

Die Arbeit gliedert sich in folgende drei Hauptteile:

\begin{enumerate}
  \item \textbf{Erarbeitung von Methoden zur Absenkung der Nutzungsschwelle von Kommandozeilen Interfaces}:
    \smallbreak
    In Literaturrecherche werden Methoden zusammengetragen und formuliert.
    Diese sind - wie im Sinne des Wortes Methode (``Weg zu etwas hin'' laut \cite{duden_methode}) - deskriptiv und beschreiben wie ein gewünschter Effekt zu erzielen sein soll.
    \begin{enumerate}
      \item \textbf{Probleme mit Kommandozeilen Interfaces}:
        \smallbreak
        Es werden zuerst Probleme, welche die Nutzung von Kommandozeilen Interfaces erschweren erläutert.
      \item \textbf{Methoden zur Absenkung der Nutzungsschwelle von Kommandozeilen Interfaces}:
        \smallbreak
        Methoden werden formuliert und wie diese zuvor geschilderte Probleme adressieren.
    \end{enumerate}

  \item \textbf{Implementation einer Anwendung auf Basis der erarbeiteten Methoden}:
    \smallbreak
    Gesammelte Methoden werden in der Implementation einer App demonstriert.
    Die App ist überschaubar komplex und adressiert das manuelle Festhalten von Arbeitsstunden.
    \begin{enumerate}
      \item \textbf{Anforderungsanalyse}:
        \smallbreak
        Zuerst werden die von der App zu erfüllenden Anforderungen erörtert.
      \item \textbf{Implementation der App unter Berücksichtigung gesammelter Methoden}:
        \smallbreak
        Es werden die Anforderungen unter Beachtung ermittelter Methoden implementiert.
    \end{enumerate}

  \item \textbf{Evaluation der implementierten Anwendung}:
    \smallbreak
    Die App wird von mit dem Anwendungsfall vertrauten Kommandozeilen Einsteigern getestet.
    Und Anhand einer Umfrage evaluiert.
    \begin{enumerate}
      \item \textbf{Umfragengestaltung}:
        \smallbreak
        Es wird die Umfrage gestaltet.
      \item \textbf{Auswertung}:
        \smallbreak
        Ergebnisse der Umfrage werden ausgewertet.
    \end{enumerate}
\end{enumerate}

\chapter{Grundlagen und Definitionen}
% [Beschreibung des Kontextes der Arbeit mit allen durch die Problemstellung tangierten Bereichen, Methoden, Theorien, Erkenntnissen, Technologien, ... ]

% TODO: CLI, how did we get here, <-> GUI

% TODO: citation needed, multiple tech definitions
\textbf{Kommandozeilen Interface}: (Engl. command-line interface) oft auch als \textbf{CLI} abgekürzt

Die Kommandozeile nimmt eine Anfrage, in Form von einer Zeile Text, entgegen und antwortet darauf.

\begin{figure}
  \centering
  \includegraphics[scale=0.5]{mullvad-status.png}
  \caption{Ändern und Überprüfen des VPN Standorts in der Linux Kommandozeile mit dem \codeinline{mullvad} CLI}
  \label{fig:mullvad-status}
\end{figure}

Die Kommandozeile existiert in zwei Ausprägungen: dem Betriebssystem CLI, auch
als Shell bekannt (siehe die Linux Shell in Abbildung \ref{fig:mullvad-status})
oder der Kommandozeile einer Anwendung (siehe die Node.js Kommandozeile in
Abbildung \ref{fig:node-calc}).

\begin{figure}
  \centering
  \includegraphics[scale=0.5]{node-calc.png}
  \caption{Durchführung einer Berechnung in der Node.js Kommandozeile}
  \label{fig:node-calc}
\end{figure}

\cite{Spolsky_2001} beschreibt neben der CLI noch zwei weitere Terminal Interfaces welche sich aus diesem entstanden sind:
\\
\textbf{Interaktive CLI}: Ein ``question and answer model'' \parencite[42]{Spolsky_2001} der Kommandozeile, wo der Nutzer mittels Fragen, auf die eine Antwort erwartet wird, entlastet wird. Mehr dazu in Kapitel \ref{chap:interactive}.
\\
\textbf{Menu-driven CLI}: auch als Ncurses CLI bekannt. Das Menü-basierte CLI ähnelt dem GUI indem

% Vergleich zu GUI
% function-orient vs object-oriented vgl. nielson1993

%%% Methoden
\chapter{Methoden zur Absenkung der Nutzungsschwelle}
% TODO: intro here

\section{Methodologie}
% [Beschreibung des geplanten Vorgehens(-modells) zur Lösung der Problemstellung; umfasst u.a.:

Die Artefakte dieses Kapitels sind die folgenden:
\begin{enumerate}
  \item Eine Auflistung von Problemen welche den Umgang mit Kommandozeilen Interfaces erschweren
  \item Eine Auflistung von Methoden, welche diese Probleme adressieren
\end{enumerate}

In Literaturrecherche sollen Probleme erschlossen werden, als auch die dafür -
möglicherweise vielfach - vorliegenden Lösungsansätze. Welche dann, unter Bezug
auf das Problem, als Methoden formuliert dokumentiert werden sollen.
\\
Die Formulierung als Methode soll dabei den Lösungsansatz so beschreiben das
dieser anwendbar ist. Dies ist u.a. erforderlich da die Methoden im nächsten
Schritt angewandt werden sollen.

\section{Probleme mit Kommandozeilen Interfaces}

Es folgen, in nicht priorisierter Reihenfolge, Probleme welche die
Nutzungsschwelle von Kommandozeilen Interfaces erhöhen, d.h. diese weniger
zugänglich.
\\
Diese bestmöglich zu adressieren sollte zu einer Absenkung der Nutzungsschwelle
und damit besser zugänglichen Applikationen führen.

\subsection{Erinnern von Kommandos}

Das Problem des Erinnern von Kommandos (engl. Command recall) beschreibt das
Nutzer sich bei Verwendung eines Programmes an dessen Namen, Kommandos und
Parameter erinnern müssen \parencite{Raskin_2008}.

\fbox{\parbox{\linewidth}{
  \label{prob:command-recall}
  \textbf{Problem~\ref{prob:command-recall}}: Es muss sich an Programmname, Kommandos und Parameter erinnert werden.
}}

Um dieses allgemeine Problem zu adressieren wird es sind kleinere Teilprobleme
aufgespalten.

\medskip

\begin{figure}
  \centering
  \includegraphics[scale=0.5]{empty-prompt.png}
  \caption{Die Shell bietet nichts an, ohne ein Programm zu kennen passiert nichts.}
\end{figure}

\cite{Gentner_1996} beschreibt auch den Fakt das es keinen einfachen Weg zum
Auffinden von Programmen gibt.
\\
Der Nutzer muss also den Namen des Programmes kennen um dieses zu nutzen.

\smallskip

Unter der Annahme den Names des Programmes zu kennen, geht es nun darum einen
Programmaufruf mit Kommandos und Parametern zusammen zu bauen.
Hierzu können verschiedene Szenarien betrachtet werden.
Der Nutzer:

\begin{enumerate}
  \item hat Kommandos und Parameter auswendig gelernt
  \item baut den Programmaufruf mit Hilfe der Dokumentation zusammen
  \item fügt den Programmaufruf (mit copy-paste) ein
  \item ruft das das Programm indirekt auf (über ein Script oder Alias)
\end{enumerate}

Diese Szenarien sind aber alle nicht ideal:

\medskip

Das 1. Szenario gilt nur mit dem Programm vertraute Nutzer. Und selbst diese
vergessen mit der Zeit. % TODO: citation needed, vergessen

\medskip

Szenario Nummer 2. ist zeitintensiv für den Nutzer. Auch gilt die Annahme das
Dokumentation existiert und das der Nutzer mit dieser umgehen kann.

\medskip

Für Szenario Nummer 3. muss der Nutzer von irgendwo kopieren, die Quelle ist
hier entweder auch die Dokumentation oder externe Portale wie Foren oder
Stack Overflow. In der Dokumentation musst der Programmaufruf auch erstmal
gefunden werden. Und die externen Portale sind für den Bau des Interfaces nicht
verlässlich.

\medskip

Szenario Nummer 4. gilt nur für erfahrene Nutzer die schon mit der Shell
vertraut sind.

\medskip

Es gibt auch Mischformen, wie das der Nutzer Kommando aber nicht Parameter kennt
und dieses nachschaut, diese sind aber in der Betrachtung zu vernachlässigen, da
die Kritiken an den Reinformen auch für sie gelten.

\subsection{Syntax und Semantik}

``Commands and associated parameters must be typed, maintaining the correct
semantic content and syntactic form.'' \parencite[184]{Westerman_1997}

\cite{Gentner_1996} beschreibt die Kommandozeile auch als sehr starr und wenig
Tolerant gegenüber imperfekter Syntax.

\fbox{\parbox{\linewidth}{
  \label{prob:syntax-semantik}
  \textbf{Problem \ref{prob:syntax-semantik}}: Die richtige Syntax and Semantik muss gewährleistet werden.
}}

\bigskip

Ein Beispiel für fehlerintolerante Syntax am Beispiel von \codeinline{grep}.
Die man page beschreibt die Syntax: \codeinline{grep [OPTION...] PATTERNS [FILE...]}.
Beim schnell-geschehenen Vertauschen von \codeinline{FILE} und \codeinline{PATTERNS}:

\begin{lstlisting}[caption={Fehlerhafte Kommando Syntax bei grep}]
grep ./file "[a-z]{3}"
/bin/grep: [a-z]{3}: No such file or directory
\end{lstlisting}

\medskip

Der Bedeutungsgehalt von Kommandos kann an diesem Beispiel dargestellt werden:

\begin{lstlisting}[caption={Kommando Semantik am Beispiel von git}]
git remote add [..]
git add remote
\end{lstlisting}

Das Kommando \codeinline{git remote} ist zum verwalten von `tracked repositories'.
\codeinline{git add} markiert eine Datei für den nächsten `commit'.
Je nach Kontext haben \codeinline{add} und \codeinline{remote} eine andere Bedeutung.
\codeinline{git remote add} fügt eine neue `repository' hinzu, \codeinline{git
add remote} markiert aber eine Datei mit dem Namen \codeinline{./remote} für den
nächsten `commit'.

\section{Gesammelte Methoden}
% Methoden beziehen sich direkt auf ein Problem welches adressiert werden soll.

\newcommand{\methbox}[2]{
  \fbox{\parbox{\linewidth}{
    \label{meth:#1}
    \textbf{Methode~\ref{meth:#1}}: #2
  }}
}
\newcommand{\methref}[1]{
  Methode~\ref{meth:#1}
}

Anders als erhofft gestaltete sich die Recherche nach Methoden zur Absenkung
der Nutzungsschwelle als schwierig da, wie bereits angesprochen, das GUI oder
eine andere Form von Nutzerinterface oft als Lösung für die Probleme von
Kommandozeilen angeführt wird.

% TODO: ...

\cite{nagarajan2018} zum Beispiel stellt auf Basis 

\subsection{Interaktive CLI}
% TODO: move this section into definitions

Als Interaktive CLI ist eine solche bekannt die dem Nutzer nach und nach Fragen
stellt, ähnlich einem Formular. Wie bei einem Formular kann neben Freitext kann
dem Nutzer mit Radiobuttons (auswählen von einer Option aus einer Liste) oder
Checkboxen (auswählen mehrerer Optionen aus einer Liste) zum auswählen. % TODO: fix

% TODO: screenshot mit Q&A format

% TODO: extend, reference sources
% bland2007design, Gentner_1996

\subsubsection{Interaktive Frage bei fehlendem Parameter}

% TODO: vgl. Command recall
Man stelle sich folgendes Szenario vor. Eine CLI hat ein Kommando welches einen Parameter erfordert.
Normalerweise erfolgt beim Vergessen der Übergabe dieses Parameters eine Fehlermeldung welche diesen Fakt schildert.

\begin{lstlisting}[caption={Fehlermeldung bei fehlendem Parameter in mullvad (VPN) CLI}]
mullvad relay set location

error: The following required arguments were not provided:
    <country>

USAGE:
    mullvad relay set location <country> [ARGS]

For more information try --help
\end{lstlisting}

Die CLI weiß aber schon was der Nutzer tun möchte und könnte anstatt der Fehlermeldung einen Prompt geben welcher den Nutzer fragt eine Parameter zu übergeben oder, je nach Kontext, einen vorzuschlagen.

\begin{lstlisting}[caption={Prompt welcher direkt nach fehlendem Parameter fragt}]
mullvad relay set location
Enter location country code:
\end{lstlisting}

Fairerweise ist zu bemerken das die mullvad CLI dies auch an anderer Stelle so handhabt, so fragt die CLI bei \codeinline{mullvad account login} nach der fehlenden Accountnummer.

% TODO: method declaration

Negativ an dieser Methode wäre das der Nutzer nicht im gleichen Maße auf die
Möglichkeit einer komplexeren Verwendung des Kommandos hingewiesen wird (so
nimmt das mullvad Kommando aus obigem Beispiel nicht nur Landes- sondern auch
Stadt- und Hostkennungen). Außerdem wird der Programmaufruf ohne den Parameter
in der Shell History gespeichert, was bedeutet bei das bei zurückgehen zu dem
Aufruf der fehlende Parameter immer noch fehlt und ergänzt oder erneut über den
Prompt übergeben werden muss. % TODO: shorten

Beide Punkte sind aber für Einsteiger eher zu vernachlässigen.

\subsection{Menü oder TUI}

% TODO: screenshot
Das Menü oder auch `Text-based user interfaces' ähneln dem GUI dadurch das der
Nutzer seine Optionen visuell präsentiert bekommt und daraus ausgewählen kann.
\\
Da es aber wie die CLI in Terminal lebt werden auch zum darstellen des Menüs
nur Textelemente verwendet. Zur Implementation wird oft die ncurses Bibliothek
verwendet.

\bigskip

Im einem Experiment von \cite{Westerman_1997} hatte das Menü-basierte
Interface, für manche Nutzergruppen, unsignifikant bessere Performance als
ein Kommandozeilen Interface. Auch wurde bei freier Wahl das Menü über alle
Nutzergruppen hinweg doppelt so häufig verwendet.

\subsection{Autovervollständigung zum Vorschlagen und Vervollständigen von Kommandos/Flaggen}

Auto-completion in der Kommandozeile beschreibt das Verhalten das bei drücken der
(normalerweise) `TAB' Taste etwas vervollständigt wird.

In der Shell funktioniert dies etwa für Programmnamen und die Pfade von Dateien:
\begin{lstlisting}[caption={Autovervollständigung in der Shell}]
wg # Nutzer drueckt TAB
wget

cat ./note # Nutzer drueckt TAB
cat ./notes.md
\end{lstlisting}

Bei Programmen kann dies, neben der Vervollständigung von Dateipfaden, auch
Kommandos und Flaggen umfassen.

\begin{figure}
  \centering
  \includegraphics[scale=0.5]{mullvad-autocomplete.png}
  \caption{Bei mullvad's CLI werden nach dem drücken von `TAB' mögliche Kommandos aufgelistet}
  \label{fig:autocomplete}
\end{figure}

% TODO: Vorschlagen beschreiben

Auto-completion adressierte beide Probleme welche das CLI mit sich bringt, durch
das Vorschlagen und Vervollständigen erleichtert das Erinnern des gesuchten
Kommandos. Und dadurch das nur valide Kommandos vorgeschlagen werden ist das
einhalten von Syntax- und Semantikregeln erleichtert. Auch profitiert der Nutzer
von Kontext-bezogenen Beschreibungen.

\subsection{Relevante Defaults}

Relevante Defaults sind nicht Kommandozeilen spezifisch, können dort aber
wichtiger sein als in grafischen Anwendungen.
\\
Dem Nutzer werden die Möglichkeiten eben nicht zur Auswahl gestellt und dieser
klickt die gewünschte Option an. Sondern ist das Auflisten der Möglichkeiten
i.d.R. ein separater Schritt, losgelöst von dem Schritt der Nutzung der Option
(z.B. werden mit \codeinline{ls -d} zuerst die Ordner aufgelistet, um sich dann
mit \codeinline{cd ORDNER} in einen von diesen hineinzubewegen, siehe auch Abbildung \ref{fig:defaults-demo}.)
\\
Es greift wieder das fundamentale Problem des `Command Recall' (vgl. Problem
\ref{prob:command-recall}). Den Schritt des Auflistens der Möglichkeiten
(ähnlich wie das Auflisten von möglichen Kommandos) ist ein extra Schritt, außer
der Nutzer kann sich an die gewünschte Option erinnern.

\begin{figure}
  \centering
  \includegraphics[scale=0.5]{defaults-demo.png}
  \caption{Der erforderte Parameter in dieser Literaturverwaltung ist einer Liste zu entnehmen}
  \label{fig:defaults-demo}
\end{figure}

\bigskip

\methbox{relevant-defaults}{Falls möglich, sollen Argumente relevanten Defaults haben.}

% TODO: elaborate

%%% Bauen
\chapter{Anforderungsanalyse}
% [Beschreibung der Erhebung, Granularisierung und Priorisierung der zu Grunde liegenden Anforderungen]
\section{Konzept}
\section{Anforderungen}
% Anforderungserhebung nur durch mich, da App nicht Fokus der Arbeit und ich User bin

\chapter{Implementation der App unter Berücksichtigung gesammelter Methoden}
\section{Fundament}

Das Fundament der App stellen die zugrundelegenden technologischen Aspekte dar,
welche nicht oder nur indirekt die Nutzbarkeit, und damit den Fokus der Arbeit
betreffen.
% ... get into tech

\section{Anwendung gesammelter Methoden}
% hier List wie die Methoden angewandt wurden

%%% Umfrage
\chapter{Umfragengestaltung}
\section{Methodologie}
\section{Ergebnisartefakte}
% [Beschreibung der Ergebnisse / Ergebnistypen, welche Sie im Rahmen der Probleml\"osung generieren / erzielen wollen, z.B. Algorithmus, Prototyp einer Software(komponente), ... ]

\chapter{Auswertung}

%%% Finish
\chapter{Zusammenfassung}
% [Aggregierte retrograde Kurzbeschreibung der Arbeit]
\section{Schlussfolgerungen}
% [Beschreibung der insgesamt zu konstatierenden Schlussfolgerungen im Zusammenhang mit der Arbeit]
\section{Limitationen}
% [Beschreibung der Ergebnisse einer kritischen Reflektion und Begr\"undung dessen, was die Arbeit nicht zu leisten vermag]
\section{Ausblick}
% [Beschreibung und Begr\"undung potenzieller zuk\"unftiger Folgeaktivit\"aten im Zusammenhang mit Ihrer Arbeit (z.B. weitere Anforderungen, Theoriebildung, ... ]

%%% possible modules
% \section{Kontext}
% \subsection{Domain}
% \subsection{Technologien}
% \subsection{Methoden und Konzepte}
% \section{...}
% \subsection{...}
% \subsection{...}
% \chapter{Anforderungserhebung und -analyse}
% \section{Nutzer- und Systemanforderungen}
% \subsection{Funktionale Anforderungen}
% \subsubsection{Obligatorisch (MUSS)}
% \subsubsection{Fakultativ (Kann)}
% \subsection{Nicht-funktionale Anforderungen}
% \subsubsection{Obligatorisch (MUSS)}
% \subsubsection{Fakultativ (Kann)}
% \section{...}
% \chapter{Konzeption \& Entwurf}
% [Beschreibung des Entwurfs auf Basis der Methodologie / der geplanten Vorgehensweise zur Probleml\"osung im Kontext der Anforderungen (i.A. der Art der Arbeit)]
% \section{Prozess}
% \section{Systemarchitektur}
% \section{Softwarearchitektur}
% \section{Schnittstellen}
% \section{Datenmanagement}
% \section{...}
% \chapter{Implementierung}
% [Beschreibung der Implementierung\footnotemark auf Basis des Entwurfs und der Methodologie / der geplanten Vorgehensweise zur Probleml\"osung im Kontext der Anforderungen. Hier ist Raum f\"ur Listings, wie z.B. das nun Folgende: Umfangreicher Quell-Code sollte in den Anhang ausgelagert werden.]
% \chapter{Test}
% [Beschreibung, wie Sie auf Basis des geplanten Testvorgehens was mit welchen Kriterien und Technologien getestet haben]
% \chapter{Darstellung und Bewertung der Ergebnisse}
% [Beschreibung der Ergebnisse aus allen voran gegangenen Kapiteln sowie der zuvor generierten Ergebnisartefakte mit Bewertung, wie diese einzuordnen sind]

% \bibliographystyle{apalike}
% \bibliographystyle{ksfh_nat} % ein anderer Stil
% \bibliography{science}
\printbibliography[
heading=bibintoc,
title={Quellenverzeichnis}
]

\newpage
\chapter{Glossar}
\begin{appendix}
\pagenumbering{Roman}
\chapter{Appendix}

\section{Quell-Code}

\section{Tipps zum Schreiben Ihrer Abschlussarbeit}

\begin{itemize}
\item Achten Sie auf eine neutrale, fachliche Sprache. Keine \glqq{}Ich\grqq{}-Form.
\item Zitieren Sie zitierf\"ahige und -w\"urdige Quellen (z.B. wissenschaftliche Artikel und Fachb\"ucher; nach M\"oglichkeit keine Blogs und keinesfalls Wikipedia.
\item Zitieren Sie korrekt und homogen.
\item Verwenden Sie keine Fu{\ss}noten f\"ur die Literaturangaben.
\item Recherchieren Sie ausf\"uhrlich den Stand der Wissenschaft und Technik.
\item Achten Sie auf die Qualit\"at der Ausarbeitung (z.B. auf Rechtschreibung).
\item Informieren Sie sich ggf. vorab dar\"uber, wie man wissenschaftlich arbeitet bzw. schreibt:
\begin{itemize}
\item Mittels Fachliteratur\footnote{Z.B. \autocite{balzert2011}, \autocite{franck2013}}, oder
\item Beim Lernzentrum\footnote{Weitere Informationen zum Schreibcoaching finden sich hier: \url{https://www.htw-berlin.de/studium/lernzentrum/studierende/schreibcoaching/}; letzter Zugriff: 13 VI 19.}.
\end{itemize}
\end{itemize}

\newpage
\thispagestyle{empty}
\noindent

\section*{Eidesstattliche Versicherung}
Hiermit versichere ich an Eides statt durch meine Unterschrift, dass ich die vorstehende Arbeit selbstst\"andig und ohne fremde Hilfe angefertigt und alle Stellen, die ich w\"ortlich oder ann\"ahernd w\"ortlich aus Ver\"offentlichungen entnommen habe, als solche kenntlich gemacht habe, mich auch keiner anderen als der angegebenen Literatur oder sonstiger Hilfsmittel bedient habe. Die Arbeit hat in dieser oder \"ahnlicher Form noch keiner anderen Pr\"ufungsbeh\"orde vorgelegen.\\
\linebreak[4]
\linebreak[4]
\linebreak[4]
\linebreak[4]
-------------------------------------------------------\linebreak[4]
Datum, Ort, Unterschrift

\end{appendix}
\end{document}
